\section{绪论}

\subsection{生态学四大学派}

\begin{description}
	\item[英美学派] 以研究植物群落的演替和创建顶极群落而著名。
	\item[北欧学派] 以注重群落结构分析为特点。
	\item[法瑞学派] 研究中首先对样地植物区系进行调查、记录和分析。
	\item[俄国学派] 植物与地学结合。
\end{description}


\section[个体生态学]{个体生态学(生物与环境)}

\subsection{环境和生态因子}
\subsubsection{环境}

	\sy{环境}分为小环境和大环境。小环境指对生物有直接影响的邻接环境,是生物可以直接改变的。大环境则指地区环境、地球环境、宇宙环境。

	生物群落带是指具有相似群落的一个区域生态系统模型,包含了具有相似非生物环境和相似生态结构的区域。

\subsubsection{生态因子}

	\sy{生态因子}就是影响生物生存的环境要素。按产生原因分为:,还可以分为密度制约因子和非密度制约因子,或分为稳定因子和变动因子。

	生态因子作用的共性:

	\begin{description}
		\item[综合性] 各种生态因子相互影响、相互制约,如光照影响大气和水;
		\item[非等价性] 各种生态因子权重不同,一定有几个是起主要作用的;
		\item[不可替代性与互补性] 各种生态因子有主次之分(非等价性),但是缺一不可。数量上的不足可以靠其他因子来弥补。如贝类在钙元素不足时会用锶构建贝壳。
		\item[限定性] 生物生长各阶段所需生态因子的种类与强度是不同的,如低温可在开花前起春化作用,在植物后续生长中却有害。
	\end{description}

	一种生态因子还能对其他生态因子产生影响,如山的阴坡和阳坡。

	利比希最小因子定律:

	布莱克曼的限制因子定律:

	谢尔福德的耐受性定律:

	对耐受性定律的补充:

	\begin{itemize}
		\item 当某个生态因子不是处于最适状态时,生物对其他生态因子的耐受范围将缩小。
		\item
	\end{itemize}

	下面将分别介绍几个生态因子的作用:

\subsection{光的生态作用}

\subsubsection{光照强度的作用}

\subsubsection{光质的作用}

红光促进植物茎叶生长,蓝光促进蛋白质合成。短波光使生长素分解,促进花青素的形成,抑制茎的伸长。

水的上层是绿藻,而褐藻、红藻分布较深。

\subsubsection{光周期现象}

植物的光周期现象参见植物生理部分。

动物的光周期现象:阿朔夫规律

异长
\subsection{温度的生态作用}

有效积温法则:\[K=N(T-T_0)\]

\subsubsection{极端温度对生物的影响}

低温

高温
\subsubsection{动物与温度}

恒温动物与变温动物相关概念:
\begin{itemize}
	\item \sy{恒温动物}(或温血动物、内温动物):靠自身产热维持恒定体温,如鸟纲、哺乳纲。
	\item \sy{变温动物}(或冷血动物、外温动物):依靠外部热源来调节体温,如鱼纲、两栖纲、爬行纲。
\end{itemize}

\sy{异温性}:恒温动物的体温偏离正常范围的现象,如冬眠\footnote{这里要区别冬眠与蛰眠,蛰眠是应对胁迫的反应。}。有异温性(会冬眠)的动物就叫\sy{异温动物}。

贝格曼规律:恒温动物在高纬度地区体型大。

乔丹规律:鱼的脊椎骨

葛洛格规律:干燥寒冷地区,动物体色较淡;潮湿温暖地区,体色较深。
\subsubsection{温度的影响}
温度影响酶的活性。\sy{范特霍夫定律}:酶促反应速度受到温度影响,温度每升高10$^{\circ}\textrm{C}$,反应速率加快到原来的$\gamma$(或者说$\textrm{Q}_{10}$倍)。$\gamma$($\textrm{Q}_{10}$)是范特霍夫系数。

\subsection{水的生态作用}

水生植物分为沉水、浮水、挺水

陆生植物分为旱生、中生植物、阳生植物

\subsection{火的生态作用}

\section{种群生态学}
\subsection{种群的概念}

种群是在一定时间内、占据一定空间的同种生物个体的集合。

\subsection{种群大小的估计}

\subsubsection{绝对密度统计方法}

\subsubsection{相对密度统计方法}

\paragraph{标记重捕法}

标记重捕法又可细分为以下三种:
\subparagraph{林可指数法(林可--彼得森法)}

这是最简单的标记重捕法,也就是高中生物学课本中的方法。一次标记一次重捕。记种群个体总数为$N$,标记个体数为$M$,重捕个体数为$n$,重捕个体中带有标志的个体数为$m$,则有:\[\hat{N}=\frac{Mn}{m}\]



\subparagraph{施夸贝尔法}

这种方法多次标记多次重捕,实际上是林可指数法的延伸。下面看一个实例(\autoref{tab:schnabel}):

\begin{table}[htbp]
	\centering
	\begin{tabularx}{\textwidth}{|c|c|c|c|X|}
		\hline
		& 捕获个体数$n_{i}$ & 带标记个体数$m_{i}$ & 新捕获个体数 & 带标记个体总数$M_{i}$ \\ \hline
		第1次捕捉 & 32    & 0      & 32     & 0       \\ \hline
		第2次捕捉 & 54    & 18     & 36     & 32      \\ \hline
		第3次捕捉 & 37    & 31     & 6      & 68      \\ \hline
		第4次捕捉 & 60    & 47     & 13     & 74      \\ \hline
		第5次捕捉 & 41    & 36     & 5      & 87      \\ \hline
		第$k$次捕捉 & $n_{k}$    & $m_{k}$     & $n_{k}-m_{k}$      & $M_{k}=(n_{k-1}-m_{k-1})+M_{k-1}$      \\ \hline
		总数    & 224   & 132    & 92     & 261     \\ \hline
	\end{tabularx}
	\caption{施夸贝尔法实例}
	\label{tab:schnabel}
\end{table}

对于每一行数据,都运用林可指数法估算出种群数量$N_{i}$,之后取加权平均数\footnote{因为捕捉个体数目多意味着更准确。},即种群数量估计值\[\hat{N}=\frac{\sum\limits_{i=1}^{k}n_{i}M_{i}}{\sum\limits_{i=1}^{k}m_{i}}\]

\subparagraph{乔利--西贝尔法}

这种方法也是多次标记多次重捕,但是因为它增加了样本被捕获的时间,所以允许种群数量在取样期间发生波动。具体方法较复杂,就不赘述了。

\paragraph{去除取样法}

在一个封闭的种群中,以相同捕捉强度分次连续捕捉,则捕捉到的个体数会减少,通过绘制累计捕捉数--每日捕捉数图像,拟合直线的横截距即为种群数量的估计值。

\subsubsection{等效种群数量}

等效种群数量($N_e$)是一个理想化的种群大小,它通常小于或等于实际的种群大小,也称为普查种群大小($N$)。当性别比例不均衡、世代重叠、繁殖成功率变化\footnote{例如只有少数个体可以成功繁殖后代的情况。}、种群规模波动或存在自然选择时,等效种群数量会下降。

下面介绍部分情况下,$N_e$的计算方法,其核心都是计算调和平均数:

\paragraph{性别比例不均衡}

在性别比例不均衡的情况下,$N_e$可以通过两倍的调和平均数来估算:
\[N_e = \frac{4N_mN_f}{N_m + N_f}\]

这里 $N_m$ 表示雄性个体的数量,而 $N_f$ 表示雌性个体的数量。因为计算的是整个种群的数量,所以公式是两倍的调和平均数。

\paragraph{种群数量波动}

如果种群在不同世代的数量分别为 $N_1, N_2, \cdots, N_t$,其中 $t$ 代表观察到的世代数,则等效种群数量
\[N_e = \frac{T}{\sum_{t=1}^{T} \frac{1}{N_t}}\]



\subsection{生物多样性的类别及计算}
\subsubsection{$\alpha$多样性}

\sy{$\alpha$多样性}(Alpha Diversity)是指一个特定区域(较小)内物种的数量及分布情况。

$\alpha$多样性指数主要包括以下几种类型:

\begin{itemize}
	\item 物种丰富度 (\textbf{Species Richness}):指的是在一个样本中物种的数量。这是一种最简单的测量方法,只考虑物种种类而不考虑物种的相对丰度。

	\item 辛普森指数 (\textbf{Simpson's Index}):该指数不仅考虑物种的数量,还考虑了每个物种个体数的分布情况。辛普森指数定义为:
	\[
	D = \sum_{i=1}^{S} \frac{n_i(n_i - 1)}{N(N-1)}
	\]
	其中 \( n_i \) 是第 \( i \) 个物种的个体数,\( N \) 是所有物种的总个体数。辛普森指数的另一个常见形式是 \( 1 - D \),其范围从 0 到 1,值越大表示多样性越高。

	\item 香农-威纳指数 (\textbf{Shannon-Wiener Index}):这是另一种常用的指数,它同时考虑了物种的丰富度和均匀度。公式如下:
	\[
	H' = -\sum_{i=1}^{S} p_i \ln(p_i)
	\]
	其中 \( p_i \) 是第 \( i \) 个物种个体数占总个体数的比例。

	\item 辛普森均匀度指数 (\textbf{Simpson's Evenness}):这是基于辛普森指数的一个变体,用来衡量物种之间的均匀程度。
\end{itemize}

\subsubsection{$\beta$多样性}

\sy{$\beta$多样性}(Beta Diversity)是用来衡量两个或多个样地之间物种组成的\textbf{差异}。

$\beta$多样性的计算可以通过多种方式实现,常见的方法包括但不限于:

\begin{itemize}
	\item \textbf{Sørensen 指数(或Dice系数)}:用于衡量两个样地间共享物种的比例。
	\[
	\textrm{Sørensen Index} = \frac{2C}{A + B}
	\]
	其中 \( C \) 表示两个样地共有的物种数量,\( A \) 和 \( B \) 分别代表两个样地的物种数量。

	\item \textbf{Jaccard 指数}:用于评估两个样地间共享物种的比率。
	\[
	\textrm{Jaccard Index} = \frac{C}{A + B - C}
	\]

	\item \textbf{Whittaker 模型}:提出了$\beta$多样性的概念,并将其定义为$\gamma$多样性(整个区域的物种多样性)与$\alpha$多样性(单一样地内的物种多样性)之差。
	\[
	\beta = \gamma - \alpha
	\]
\end{itemize}

$\beta$多样性是生态学研究中的一个重要概念,因为它帮助我们理解物种如何随着环境条件的变化而在不同区域分布。对于保护生物学、生态学研究以及资源管理等领域而言,$\beta$多样性提供了一个重要的视角,帮助科学家们更好地理解生物多样性格局及其动态变化。

\subsubsection{$\gamma$多样性}

\sy{$\gamma$多样性}(Gamma Diversity)是指一个较大区域或生态系统内总的物种多样性。$\gamma$多样性不仅考虑了物种的数量,还包括了各个物种在整个区域中的分布情况。

$\gamma$多样性的计算较为直观,可以通过简单汇总各个样地的物种来完成。假设有 \( k \) 个样地,每个样地 \( j \) 中的物种集合为 \( S_j \),那么整个区域的物种集合 \( S \) 可以表示为所有样地物种集合的并集:
\[
S = \bigcup_{j=1}^{k} S_j
\]

$\gamma$多样性的物种丰富度 \( R \) 即为集合 \( S \) 中元素数:
\[
R = |S|
\]

$\gamma$多样性也可以通过统计整个区域内的所有物种来计算。例如,假设一个区域由三个样地组成,每个样地的物种列表如下:
\[
S_1 = \{\textrm{A, B, C}\}, \quad S_2 = \{\textrm{B, C, D}\}, \quad S_3 = \{\textrm{C, D, E}\}
\]

则整个区域的物种集合为:
\[
S = \{\textrm{A, B, C, D, E}\}
\]

因此,$\gamma$多样性中的物种丰富度 \( R \) 为 5。

\subsection{种群的动态}

阿利氏定律:动物有一个最适的种群密度,过密或过疏都可能产生不利影响。

\subsubsection{生存曲线}

生存曲线是以年龄为横轴、存活个体数的对数值为纵轴绘制的曲线。生存曲线分为I、II、III型。

\subsubsection{生命表}

生命表是

生命表分为静态生命表和动态生命表。

\subsubsection{种群增长}

\begin{description}
	\item[瞬时增长率]
	\item[内禀增长率]
	\item[周限增长率] description
\end{description}

\subsubsection{种群数量的调节}

解释种群数量调节的理论,有的强调外因,即环境因素的影响;有的强调内因,即种群内部变化的影响。

